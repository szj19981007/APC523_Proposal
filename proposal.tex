\documentclass{article}
\usepackage{amsmath}
\usepackage{graphicx}
\usepackage{geometry}
\geometry{left = 2.7cm, right = 2.7cm}
%\usepackage{float}
\usepackage{cite}
\usepackage{hyperref}

\title{A Parallel Solver for Stiff Chemistry}
\author{Zhiyu Shi, Zijian Sun}
\date{}
\begin{document}

\maketitle
\section{Background}

\qquad Computational fluid dynamics (CFD) simulation plays an important role in the desgin of applications such as internal combustion engines and gasturbines.
The low cost for a run of CFD simulation gains it an edge over traditional engine experiments.
However, some problems on CFD simulation method such as the accuracy and the robustness need to be further improved.
Modeling and solving the chemical reactions in combustion could be the most tricky part in CFD simulation for engines.
To meet the demands on accuracy, the reaction mechanism has to include enough species and reactions,
whereas, a too comprehensive mechanism could make the simulation computationally expensive.
For example, the detailed combustion mechanism of n-heptane, a commonly used surrogate fuel of diesel,
includes 561 species and 2539 reactions\cite{bib4}, 
which is too big that generally its skeleton mechanism instead of the comprehensive mechanism is used for CFD simulation.
Another challenge in solving the chemical reactions is dealing with the stiff chemistry. 
The stiffness in a reaction mechanism is caused by the highly reactive radicals and their short timescales\cite{bib6}.
In solving the stiff reactions, expensive implicit ODE solvers such as VODE\cite{bib7} has to be applied
to maintain the convergency of iterations with a reasonablly large time step. 

To improve the computational efficiency of chemical reactions in combustion process,
 parallel computation is used in most commerical CFD packages such as Ansys Fluent\cite{fluent} and CFX\cite{cfx}.
The basic idea for the parallel computation in these softwares is spliting the computation domain into different regions
 and distribute them to the corresponding CPUs. 
In essence, their parallel computation is on the grids aspect instead of the chemical solver itself.
This method could have a low efficiency when the speeds of convergence in different regions varies significantly.
So in this project, we aim to develop a 0D chemical solver based on the open-source software Cantera\cite{cantera}
 and try to integrate the different iteration methods
 into the solver to improve the computational efficiency in calculating the stiff mechanism. 
\section{Problem Statement}
\qquad In a chemical solver, the elementary reactions for combustion is represented in the general form as
\begin{equation}
    \sum_{k=1}^Kv'_{ki}\chi_k\rightleftharpoons \sum_{k=1}^Kv''_{ki}\chi_k \qquad (i = 1, \cdots, I),
\end{equation}
where $v'_{ki}$ and $v''_{ki}$ are the stoichiometric coefficients for the forward and backward reactions,
 the subscript $ki$ means they are the stoichiometric coefficients of the $k$ th species in the $i$ th reaction.
 $\chi_k$ is the chemical symbol for the $k$ th species in combustion system.
Then, the production rate for $k$ th species is given by
\begin{equation}
    \frac{d[X_k]}{dt} = \sum_{i=1}^I (v''_{ki}-v'_{ki})q_i \qquad (k = 1, \cdots, K).
    \label{eq:2}
\end{equation}
In equation\eqref{eq:2}, $X_k$ is the mole concentration of the $k$ th species, 
and $q_i$ is the rate of progress variable for the $i$ th reaction which is calculated by equation\eqref{eq:3}
\begin{equation}
    q_i = k_{fi}\prod^K_{k=1}\left[X_k\right]^{v'_{ki}}-k_{ri}\prod^K_{k=1}\left[X_k\right]^{v''_{ki}},
    \label{eq:3}
\end{equation}
where $k_{fi}$ and $k_{ri}$ are the reaction rate constant for the forward and reverse reaction.
The forward reaction rate constant $k_{fi}$ is generally assumed to have the following Arrhenius temperature dependence:
\begin{equation}
    k_{fi} = A_iT^{\beta_i}\exp\left(-\frac{E_i}{R_cT}\right).
\end{equation}
In the thermo system, the reverse reaction rate constant $k_{ri}$ is related to the forward rate constant
through the equation
\begin{equation}
    k_{ri} = \frac{k_{fi}}{K_{ci}},
\end{equation}
where $K_{ci}$ is the equilibrium constant which is derived from the thermo data of species.
\begin{equation}
    K_{ci}= \exp\left(\frac{\Delta S_i^o}{R}-\frac{\Delta H_i^o}{RT}\right)\left(\frac{P_{atm}}{RT}\right)^{\sum_{k=1}^K(v''_{ki}-v'{ki})}
\end{equation}
Equation \eqref{eq:2} is the 
\section{Related Works}
\section{Project Targets}


\bibliographystyle{unsrt}

\bibliography{proposal.bib}

\end{document}